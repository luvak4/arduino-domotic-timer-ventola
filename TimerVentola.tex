\documentclass[DIV=24]{scrartcl}
\usepackage[italian]{babel}
\usepackage{microtype}
\usepackage{tikz}
\usetikzlibrary{positioning}
\usetikzlibrary{shapes}
\usetikzlibrary{arrows}
\usepackage{fontspec}

\usepackage{rotating}
\usepackage{listings}
\usepackage{graphicx}

\input{files/01impostazioni.tex}

\newcommand{\oreProgrammate}{\color{red}{oreProgrammate}}
\newcommand{\tempoAspegnimento}{\color{blue}{tempoAspegne}}
\newcommand{\conteggioAttivo}{\color{green!60!black}{conteggioAttivo}}

\title{Timer per ventilatore a pale}
\author{Luca Valcavi}
\date{\today}

\begin{document}
    \setmainfont{DejaVu Sans Mono}
    \maketitle
    \tableofcontents
    \begin{abstract}
    	Modifiche versione 4:
    	\begin{itemize}
    		\item correzione errori nella base dei tempi (Pa=Qa-Xa e Pb=Qb-Xb)
    		\item impostate ore 2,4,6,8
    		\item spegnimento del led sulla porta 13 dopo un lampeggio iniziale quando si collega il circuito
    	\end{itemize}
    	\bigskip
        Il circuito è un timer programmabile per l'accensione e lo spegnimento automatico della pala del ventilatore. 
        \begin{itemize}
        \item Il timer si attiva premendo il pulsante. 
        \item L'indicazione del tempo programmato è visualizzata da quattro LED luminosi che si accendono progressivamente ad ogni pressione del pulsante.
        \item Alla quarta attivazione il tempo impostato riparte da zero.        
        \item Una pressione prolungata del pulsante permette di spegnere il timer se risulta già acceso. Se  spento, la stessa prolungata accensione ne determina l'attivazione di un tempo predefinito pari a tre LED accesi.
        \end{itemize}
    \end{abstract}
    \section{Schemi a blocchi}    
    \subsection{loop}
    \begin{center}
        \input{files/loop.tex}
    \end{center}
    \subsection{Orologio}
    \begin{center}
        \input{files/orologio.tex}
    \end{center}    
	\subsection{chkTEMPO}
    \begin{center}
        \input{files/chktempo.tex}
    \end{center}
    \subsection{setLEDs}
    \begin{center}
        \input{files/setLeds.tex}
    \end{center}       
    \subsection{chkPremuto}
    \begin{center}
        \input{files/chkpremuto2.tex}
    \end{center}    
    \subsection{chkRilasciato}
    \begin{center}
        \input{files/chkrilasciato.tex}
    \end{center}
    \subsubsection{FunzioneA}
        \lstinputlisting[style=Arduino,firstline=291,lastline=343]{code/TimerVentilatore4/TimerVentilatore4.ino}        
    \subsubsection{FunzioneB}
        (non usata)
    \subsubsection{FunzioneC}
        \lstinputlisting[style=Arduino,firstline=221,lastline=227]{code/TimerVentilatore4/TimerVentilatore4.ino}        
    \subsubsection{FunzioneD}
        (non usata)
    \section{Codice completo}
        \lstinputlisting[style=Arduino,firstline=1,lastline=350]{code/TimerVentilatore4/TimerVentilatore4.ino}
    \section{Schema elettrico}
    \includegraphics[width=16cm]{images/schema.png}
    \section{PCB}
    \includegraphics[width=16cm]{images/pcb.png}
\end{document}
